\documentclass[preview]{standalone}
\usepackage[english]{babel}
\usepackage{amsmath}
\usepackage{amssymb}
\begin{document}
\begin{align*}
= 6.86 \text{ dB}
\end{align*}
\end{document}
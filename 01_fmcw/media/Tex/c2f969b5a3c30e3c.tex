\documentclass[preview]{standalone}
\usepackage[english]{babel}
\usepackage{amsmath}
\usepackage{amssymb}
\begin{document}
\begin{center}
$v_{max} = \frac{\lambda}{4 \cdot T} = \ $18,888.99 $\frac{m}{s}$
\end{center}
\end{document}
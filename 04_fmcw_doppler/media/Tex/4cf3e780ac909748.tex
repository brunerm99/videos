\documentclass[preview]{standalone}
\usepackage[english]{babel}
\usepackage{amsmath}
\usepackage{amssymb}
\begin{document}
\begin{center}
Here we have the coherent processing interval, $T_{CPI}$, which is comprised of $M$ pulse repetition intervals, $T_{PRI}$. This gives us our velocity resolution, 

\begin{equation}
    \Delta f = \frac{1}{M T_{PRI}}
\end{equation}

From this, we can define the pulse train LFM as

\begin{equation}
    s(t) = \sum_{m=0}^{M-1} s_p(t - m T_{PRI})
\end{equation}

where $s_t$ is the single LFM pulse waveform with duration, $T_c$,

\begin{equation}
    s_p = \begin{cases} 
      \exp{j \pi \frac{B}{2 T_c} t^2} & 0 \le t \le T_c \\
      0 & \text{otherwise}
   \end{cases}
\end{equation}

For a target with velocity, $v$, and range, $r$, the equation for the beat signal becomes

\begin{equation}
    b(t) = a \exp{\left[ j 2 \pi \frac{2 B r}{T_c c} (t - m T_{PRI}) \right]} \exp{\left[ j 2 \pi \frac{2 f_c v}{c} t \right]}
\end{equation}

for $(m-1) T_{PRI} \le t \le m T_{PRI}$ and $0 \le m \le M$.

Then if you sample the $M$ pulse repitions with a sampling rate of $T_s$, you get the 2D sampled beat signal,

\begin{equation}
    b[l, m] = a \exp{\left[ j 2 \pi \left( \frac{2 f_c v}{c} + \frac{2 B r}{T_c c} \right) l T_s \right]} \exp{\left[ j 2 \pi \frac{2 f_c v}{c} m T_{PRI} \right]}
\end{equation}

Then taking the FFT on $b[l, m]$, you get

\begin{equation}
    B[p, k] = \frac{1}{\sqrt{N_z M}} \sum_{l=0}^{N_z-1} \sum_{m=0}^{M-1} b[l, m] \exp{\left[ -j 2 \pi \left( \frac{lp}{N_z} + \frac{mk}{M} \right) \right]}
\end{equation}

where the indices $p$ and $k$ are

\begin{equation}
    p = \left( \frac{2 f_c v}{c} + \frac{2 B r}{T_c c} \right) T_c \\
    k = \frac{2 f_c v}{c} M T_{PRI}
\end{equation}
\end{center}
\end{document}